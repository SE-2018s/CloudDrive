\chapter{软件质量特性}

\section{适应性}

本云盘可以在移动平台、桌面平台访问,只需要一个通用的浏览器即可,覆盖了基本所有的使用场景

\section{可用性}

本云盘可提供99.9\%的可用性,即连续运行一年时间里期望中断时间是8.76小时:这依赖于底层文件系统的高可用性与操作系统的稳定性以及网络链接的稳定性

\section{安全性}

本云盘系统使用加密链接,防止技术层面上的信息泄露。同时有着完善的权限管理,防止用户绕过权限对不可见、不可修改的文件进行操作。

\section{正确性}

云盘建立在TCP链接上,已有基础的正确性保证的同时还会对数据传输加上CRC校检码,保证数据传输的正确性

\section{灵活性}

本云盘能适应多种不同的操作、应用场景,满足不同用户的操作习惯:
在线解压、压缩,多种方式复选,鼠标拖拽复选、移动

同时本云盘还提供上传、下载的断点续传功能

\section{交互工作能力}

本云盘的交互方式繁多但易用:

1. 复选框

2. 右键操作与功能按钮操作

3. 多种排列方式显示

4. 缩放

5. 多个标签页

允许用户灵活的组合各个功能来使用本云盘,实现复杂操作

\section{可维护性}

本云盘系统尽最大努力实现机制与策略分离,各个模块之间低耦合,方便后期调整策略、维护

\section{可移植性}

本云盘系统的客户端为网页,从而可以在任意平台上使用。服务器端则能适配大多数Linux平台

\section{可靠性}

本云盘系统对绝大多数应用场景中会出现的异常情况做了包容处理,以及多个用户对公共文件的同时处理的包容,通过跨磁盘的存储冗余机制,绝对地降低了系统宕机的可能性,使可靠性得到保证
 
\section{可重用性}

本云盘的各个模块对其他模块的依赖性都被尽量降低,使得其可以方便的重用到其他项目中

\section{鲁棒性}

本云盘除了对请求进行校验,还将对多种网络攻击手段进行预防,避免异常行为的发生影响系统运行和用户使用

\section{可测试性}

本云盘系统的各个模块都经过了完善的测试,保证能经受用户的各种操作而稳定运行
