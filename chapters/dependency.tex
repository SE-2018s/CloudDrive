\chapter{依赖关系}

{

\begin{table}[htbp]
\color{red}
\centering
\caption{功能需求的依赖关系表} \label{tab:simpletable}
\begin{tabular}{|c|c|c|}
    \hline
    依赖部分 & 被依赖部分 & 解释  \\
    \hline
    R..USER.LOGIN.002 & R..USER.LOGIN.001 & 显示初始登录界面之后才可以验证密码 \\
    \hline
    R..USER.LOGIN.003 & R..USER.LOGIN.001 & 显示初始登录界面之后才可以注册 \\
    \hline
    R..USER.LOGIN.004 & R..USER.LOGIN.001 & 显示初始登录界面之后才可以重置密码 \\
    \hline
    R..USER.LOGIN.005 & R..USER.LOGIN.002 & 用户登录之后才可以退出 \\
    \hline
    R..USER.MEMBERSHIP.001 & R.USER.LOGIN.002 & 用户登陆后在可以开通会员 \\
    \hline
    R..USER.MEMBERSHIP.002 & R.USER.MEMBERSHIP.001 & 用户开通会员后才可以续费 \\
    \hline
    R..USER.MEMBERSHIP.003 & R.USER.MEMBERSHIP.001 & 用户消费后才可以有优惠活动 \\
    \hline
    R..FILE & R..USER.LOGIN.002 & 所有的文件操作(除了游客\\ &  &操作之外)都需要用户登录 \\ 
    \hline
    R..SHAREFOLDER & R..USER.LOGIN.002 & 所有的共享文件夹操作(除了\\ &  &游客操作之外)都需要用户登录\\
    \hline
    R..USER.SOCIAL.002 & R..USER.SOCIAL.001 & 用户在搜索用户信息后在可添加好友 \\
    \hline
    R..FILE.BASIC.002 & R..FILE.BASIC.001 & 用户下载之前该文件/文件夹要被上传 \\
    \hline
    R..FILE.BASIC.012 & R..FILE.BASIC.001 & 用户加速上传时需先开始上传文件 \\
    \hline
    R..FILE.BASIC.012 & R..FILE.BASIC.002 & 用户加速下载时需先开始下载文件 \\
    \hline
    R..FILE & R..FILE.HIGH.003 & 对于加密的文件或文件夹,\\ &  & 用户操作之前还要输入密码进行确认\\
    \hline
    R..FILE.HIGH.011 & R..FILE.HIGH.010 & 审核系统需要部分依赖举报功能 \\
    \hline
    
\end{tabular}
\note{表中的R..XXX等均指R.XYZ.CLOUDSTORAGE.XXX}
\end{table} 
}

\begin{table}[htbp]
\centering
\caption{性能需求的依赖关系表} \label{tab:simpletable}
\begin{tabular}{|c|c|c|}
    \hline
    依赖部分 & 被依赖部分 & 解释  \\
    \hline
    网络带宽、延迟 & 网络硬件环境, & 高带宽、低延迟,要求极高的网络环境 \\
    & 服务器网络结构 &  \\
    \hline
    用户存储空间 & 磁盘硬件大小 & 足够的用户存储空间需要足够的磁盘容量、数量 \\
    \hline
    存储冗余 & Ceph文件系统 & 可靠高效的存储冗余需要利用Ceph文件系统 \\
    \hline
    客户端的运行 & 用户终端硬件基础 & 客户端需要至少200kB/s的带宽,\\ & & 以及1GB以上的内存 \\
    \hline
    
\end{tabular}
\end{table} 




\begin{table}[htbp]
\centering
\caption{外部接口的依赖关系表} \label{tab:simpletable}
\begin{tabular}{|c|c|c|}
    \hline
    依赖部分 & 被依赖部分 & 解释  \\
    \hline
    客户端的运行 & 用户终端软件基础 & 客户端需要运行在主流操作系统上:\\ & & 如:Linux、MacOS、Windows \\ & & 客户端需要运行在主流浏览器中: \\ & & 如:Chrome、Edge、Firefox \\
    \hline
    数据管理系统 & Mysql & 优质的开源DBMS \\
    \hline
    服务器操作系统 & CentOS & 以稳定出名的开源操作系统\\
    \hline
    分布式文件系统 & Ceph & 以稳定高效高可用出名的分布式文件系统\\
    \hline
    HTTP服务器 & Tomcat & 以便捷高效出名的Apache项目 \\
    \hline
    后端框架语言 & Java & 使用量最多的编程语言\\
    \hline
    服务器架构 & x64集群或云主机 & 需要庞大数量的服务器\\
    \hline
    通讯接口 & HTTP/HTTPS,TCP/IP协议 & 网络应用必备的协议 \\
    \hline
    
\end{tabular}
\end{table} 



    